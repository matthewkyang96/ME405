\hypertarget{index_Introduction}{}\section{Introduction}\label{index_Introduction}
This code measures the position of the motor, and returns it as a value. It will account for both overflow and underflow in either direction. The period of the timer module should be set at 65,535 to return accurate results. The user will define two input pins and a timer with two timer channels. When changing position and executing the read method, the new location will be returned. The zero method will return the position counter back to zero.\hypertarget{index_Purpose}{}\section{Purpose}\label{index_Purpose}
The drivers purpose is to set up an encoder for a motor.\hypertarget{index_Usage}{}\section{Usage}\label{index_Usage}
The code used to create an object and use it entails\+: \begin{DoxyVerb}  For pin C6 and C7 with timer 8

      tim8 = pyb.Timer(8, prescaler=1, period=65535)

      pinC6 = pyb.Pin (pyb.Pin.board.PC6, pyb.Pin.IN)

      pinC7 = pyb.Pin (pyb.Pin.board.PC7, pyb.Pin.IN)

  For pin B6 and B7 with timer 4

      pinB6 = pyb.Pin (pyb.Pin.board.PB6, pyb.Pin.IN)

      pinB7 = pyb.Pin (pyb.Pin.board.PB7, pyb.Pin.IN)

      tim4 = pyb.Timer(4, prescaler=1, period=65535)
\end{DoxyVerb}
\hypertarget{index_Testing}{}\section{Testing}\label{index_Testing}
The code was tested by spinning the motor and executing the read function The read function determines the total position that the motor rotated. The encoder accounts for the overflows when the position of the motor passes 65535 and -\/65535.\hypertarget{index_Limitations}{}\section{Limitations}\label{index_Limitations}
Running the clock at 20,000 Hz and a motor constant having 1000 counts per revolution, the motor needs to spin faster than 655,340 revolutions per second to generate an inaccurate reading. The bounds of the motor position were not tested.\hypertarget{index_Location}{}\section{Location}\label{index_Location}
The location of the source code at \href{http://wind.calpoly.edu/hg/mecha11}{\tt http\+://wind.\+calpoly.\+edu/hg/mecha11} under lab1 